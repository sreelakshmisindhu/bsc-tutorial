\documentclass[10pt, aspectratio=169]{beamer}

\usetheme{ugoe}
% Keeping graphics nice and tidy in a subfolder
\graphicspath{{./images/}{./}{/home/chris/Documents/PhD/Presentations/common_images/}}


% In general, you should encode all files using UTF-8
\usepackage[utf8]{inputenc}
\usepackage[T1]{fontenc}

% Set the math mode font to sans-serif
\let\mathrm\mathsf

% Use this for better font scaling (esp. if you want to use \tiny)
\usepackage{lmodern}
\usepackage{bm}
\usepackage{exscale}
\usepackage[normalem]{ulem}

\usepackage{multirow}
\usepackage{booktabs}
%\usepackage{ctable}
\usepackage{ragged2e}

\usepackage[yearmonthsep={-}, monthdaysep={-}, style=default]{datetime2}

\usepackage{xspace}
\usepackage{commath}
\usepackage{amsmath,amssymb}

\usepackage{siunitx}
\sisetup{separate-uncertainty = true, multi-part-units=single, list-final-separator = {,\;}, list-separator = {,\;}, list-pair-separator = {,~\mathrm{and}~}, exponent-product=\cdot}
\DeclareSIUnit{\lightspeed}{c}
\DeclareSIUnit{\electronvolt}{{e}\kern-0.08em{V}}
\DeclareSIUnit{\evmass}{\electronvolt\per\lightspeed\squared}
\DeclareSIUnit{\evmom}{\electronvolt\per\lightspeed}

\usepackage{tabularx}
\newcolumntype{C}[1]{%
    >{\setlength{\baselineskip}{0.8\baselineskip}
    \centering\arraybackslash}m{#1}
    }
\newcolumntype{L}[1]{%
    >{\setlength{\baselineskip}{0.8\baselineskip}
    \raggedright\arraybackslash}m{#1}
    }
\newcolumntype{R}[1]{%
    >{\setlength{\baselineskip}{0.8\baselineskip}
    \raggedleft\arraybackslash}m{#1}
    }
\newcolumntype{Y}{
    >{\setlength{\baselineskip}{0.8\baselineskip}
    \centering\arraybackslash}X
    }

\usepackage[compat=1.1.0]{tikz-feynman}

\usepackage{enumitem}
\setlist[itemize]{font=\color{ugoelogodark}}
\setlist[itemize,1]{label=$\bullet$}
\setlist[itemize,2]{label=--}
\setlist[enumerate]{font=\color{ugoelogodark}}
\setlist[enumerate,1]{label=\arabic*.}

%\usetikzlibrary{calc}

\newcommand\Overline[2][1pt]{%
  \begin{tikzpicture}[baseline=(a.base)]
    \node[inner xsep=0pt,inner ysep=1.5pt] (a) {$#2$};
    \draw[line width= #1] (a.north west) -- (a.north east);
  \end{tikzpicture}
}
\newcommand\textoverline[2][0.05em]{%
  \kern-0.05em
  \begin{tikzpicture}[baseline=(a.base)]
    \node[inner xsep=0pt,inner ysep=1.5pt] (a) {#2};
    \draw[line width= #1] ([yshift=0ex]a.north west) -- ([yshift=0ex]a.north east);
  \end{tikzpicture}
  \kern-0.37em
}
\newcommand{\textboldoverline}[2][0.07em]{
  \kern-0.35em
  \begin{tikzpicture}[baseline=(a.base)]
    \node[outer xsep=0pt, inner xsep=0pt,inner ysep=1.5pt] (a) {#2};
    \draw[line width= #1] (a.north west) -- (a.north east);
  \end{tikzpicture}
  \kern-0.35em
}

%\newcommand*{\textboldoverline}[1]{$\bm\bar{\mbox{#1}}$}
%\newcommand*{\textoverline}[1]{$\overline{\mathrm{#1}}$}
\newcommand*{\highl}[1]{\textbf{\color{ugoelogodark}#1}}

\newcommand{\dzero}      {D\O\xspace}
\newcommand{\cdf}        {CDF\xspace}
\newcommand{\uubar}      {\mbox{u\textoverline{u}}\xspace}
\newcommand{\ddbar}      {\mbox{d\textoverline{d\kern0.05em}\kern-0.05em}\xspace}
\newcommand{\ccbar}      {\mbox{c\kern-0.02em\textoverline{\kern0.02em{}c\kern0.02em}\kern-0.02em}\xspace}
\newcommand{\ssbar}      {\mbox{s\kern-0.02em\textoverline{\kern0.02em{}s\kern0.02em}\kern-0.02em}\xspace}
\newcommand{\ttbar}      {\mbox{t\kern-0.02em\textoverline{\kern0.05em{}t}}\xspace}
\newcommand{\bbbar}      {\mbox{b\kern-0.05em\textoverline{\kern0.05em{}b}}\xspace}
\newcommand{\ccbarbold}  {\mbox{\textbf{c\kern-0.02em\textboldoverline{\kern0.02em{}c\kern0.02em}\kern-0.02em}}\xspace}
\newcommand{\ttbarbold}  {\mbox{\textbf{t\kern-0.05em\textboldoverline{\kern0.05em{}t}}}\xspace}
\newcommand{\bbbarbold}  {\mbox{\textbf{b\kern-0.05em\textboldoverline{\kern0.05em{}b}}}\xspace}

\newcommand{\wjets}      {\mbox{W\,+\,4\;jets}\xspace}
\newcommand{\pttbar}     {\mbox{p\textsubscript{\ttbar}}\xspace}
\newcommand{\pwjets}     {\mbox{p\textsubscript{W\,+\,4\;jets}}\xspace}
\newcommand{\ljets}      {\mbox{$\ell$\,+\,jets}\xspace}
\newcommand{\ejets}      {\mbox{e\,+\,jets}\xspace}
\newcommand{\mujets}     {\mbox{$\mu$\,+\,jets}\xspace}
\newcommand{\ttjets}     {\mbox{\ttbar{}\,+\,jets}\xspace}
\newcommand{\ttljets}    {\mbox{\ttbar{}\,+\,light jets}\xspace}
\newcommand{\ttlight}    {\mbox{\ttbar{}\,+\,light}\xspace}
\newcommand{\ttc}        {\mbox{\ttbar{}\,+\,c}\xspace}
\newcommand{\ttb}        {\mbox{\ttbar{}\,+\,b}\xspace}
\newcommand{\wt}         {\mbox{Wt}\xspace}

\newcommand{\ttx}        {\mbox{\ttbar{}X}\xspace}
\newcommand{\ttv}        {\mbox{\ttbar{}V}\xspace}
\newcommand{\thiggs}     {\mbox{tH}\xspace}
\newcommand{\tth}        {\mbox{\ttbar{}H}\xspace}
\newcommand{\hbb}        {\mbox{H\,$\rightarrow$\,\bbbar}\xspace}
\newcommand{\thbb}       {\mbox{\thiggs{}(\hbb)}\xspace}
\newcommand{\tthbb}      {\mbox{\tth{}(\hbb)}\xspace}
\newcommand{\tthgam}     {\mbox{\tth{}(H\,$\rightarrow$\,$\gamma\gamma$)}\xspace}
\newcommand{\ttbb}       {\mbox{\ttbar{}\,+\,\bbbar}\xspace}

\newcommand{\thiggsbold} {\mbox{\textbf{tH}}\xspace}
\newcommand{\tthbold}    {\mbox{\ttbarbold{}\textbf{H}}\xspace}
\newcommand{\hbbbold}    {\mbox{\textbf{H}\,$\boldsymbol{\rightarrow}$\,\bbbarbold}\xspace}
\newcommand{\thbbbold}   {\mbox{\thiggsbold{}\textbf{(}\hbbbold{}\textbf{)}}\xspace}
\newcommand{\tthbbbold}  {\mbox{\tthbold{}\textbf{(}\hbbbold{}\textbf{)}}\xspace}

\newcommand{\thiggstitle}{\mbox{tH}\xspace}
\newcommand{\tthtitle}   {\mbox{t\kern-0.02em$\overline{\kern0.05em{}\mathrm{t}}$H}\xspace}
\newcommand{\hbbtitle}   {\mbox{H\,$\rightarrow$\,b\kern-0.05em$\overline{\kern0.05em{}\mathrm{b}}$}\xspace}
\newcommand{\tthbbtitle} {\mbox{\tthtitle{}(\hbbtitle)}\xspace}

\newcommand{\pt}         {\mbox{$p_{\kern0.1em\mathrm{T}}$}\xspace}
\newcommand{\mathpt}     {p_{\kern0.1em\mathrm{T}}}


\AtEndPreamble{\hypersetup{
pdftitle={},
pdfsubject={},
pdfauthor={},
pdfkeywords={}
}}

\title{\Large\color{ugoelogodark} BSc Intro 2022 Python Tutorial}
\shorttitle{Python Tutorial}
\author[]{Wael Alkakhi, Ishan Pokharel, Chris Scheulen, Sreelakshmi Sindhu}
\institute[Univ. of G\"ottingen]{II.~Physikalisches Institut, Georg-August-Universit\"at G\"ottingen}
\date{2022-03-09}

\def\insertlogos{
%   \centering
%   \includegraphics[height=1.1cm]{logos/BMBF-Gef-Logo-gefoerdert.jpg}%
%   \hspace{0.5cm}%
%   \includegraphics[height=1.1cm]{logos/FSP_ATLAS.png}%
%   \end{minipage}
}

\begin{document}

\begin{frame}
\titlepage
\end{frame}

\begin{frame}{Short Demo of Python}
  \begin{itemize}
  \item
    The interpreter: A simple way of testing things out
  \item
    Variables in Python
  \item
    Indentation or how I learned to stop worrying and love C++'s Semicolons
  \item
    Control flow: The ifs and loops of programming
  \item
    Using functions and classes in Python
  \item
    Multiple files and imports
  \end{itemize}

  \begin{center}
    \highl{Now} it's your turn! We've prepared some exercises for you in the following. Try them out! \\[0.5\baselineskip]
    If anything is unclear or if you are stuck: Ask us. \\[0.5\baselineskip]
    Maybe also try using \texttt{git} to version control your code. That will also allow us to look at it together if you are stuck somewhere!
  \end{center}
\end{frame}

\begin{frame}{Python Virtual Environments}
  \begin{itemize}
  \item
    On \texttt{lxplus} (or most other clusters), you are presented with a certain selection of Python modules
    \begin{itemize}
    \item
      Load as follows (on \texttt{lxplus}): \\
      \quad \texttt{\$ setupATLAS; lsetup "python <your version>"}
    \item
      Look-up versions as discussed yesterday: \\
      \quad \texttt{\$ showVersions python}
    \end{itemize}
  \item
    In general: OK, but often it would be nice to install some custom modules (e.g.\ \texttt{tensorflow} for Machine Learning), choose different versions (e.g.\ for \texttt{numpy}), or have multiple versions of Python for different purposes
  \item[$\Rightarrow$]
    Python virtual environments are your friend!
  \item
    For setting them up:
    \begin{itemize}
    \item
      Create venv: \texttt{\$ python{3} -m venv <path/to/venv>}
    \item
      Activate venv: \texttt{\$ source <path/to/venv>/bin/activate} \\
      Notice how \texttt{(<venvname>)} should now be in front of your shell prompts!
    \item
      Add modules: \texttt{\$ pip install \{-U\} <module>} (\texttt{-U} for updates)
    \end{itemize}
  \item
    Try it out! Build yourself a Python3 venv with the following modules (for tomorrow/Friday!): \\
    \texttt{ruamel.yaml}, \texttt{matplotlib}, \texttt{uproot}, \texttt{pandas}, \texttt{tensorflow}, \texttt{scikit-learn}, \texttt{tables}, \texttt{pydot} \\
    \highl{But:} Start by updating \texttt{pip}: \texttt{\$ pip install -U pip}
  \end{itemize}
\end{frame}

\begin{frame}{Python Interpreter and Macros}
  \begin{enumerate}
  \item
    Open the python interpreter and print out `Hello World!'
  \item
    Now, do the same thing from a \texttt{macro.py} file
  \item
    Probably, you needed to use the following for the last task: \\
    \quad \texttt{\$  python macro.py} \\
    There is also another way involving a so-called \emph{shebang}. Try it out with that!
  \item
    Use your file to print out the integers 1 to 20
  \item
    Now only print the even integers from 1 to 20
  \item
    Can you store these integers in a file?
  \end{enumerate}

  \textit{Hints:}
  \begin{itemize}
  \item
    Python has a concept similar to the main-method in C++. Try the following for that: \\
    \quad \texttt{if \_\_name\_\_ == "\_\_main\_\_":}
  \item
    To make a file executable with Linux, you also have to change it's permissions. For that you can use \texttt{chmod +x <executable>}
  \item
    Try out what the following command does in the interpreter: \\
    \quad \texttt{range(20)}
  \end{itemize}
\end{frame}

\begin{frame}{Lists and Dictionaries}
  \begin{enumerate}
  \item
    Write your name, age, and level of study in a \texttt{dictionary}
  \item
    Ask the other people in the room about their corresponding info and compile a \texttt{list} of all people
  \item
    Print the first names of all people in the room separated by commas
  \item
    Save the information of everybody in the room in a text file
  \end{enumerate}

  \textit{Hints:}
  \begin{itemize}
  \item
    The structure of a dictionary is \texttt{\{<key>:\ <value>, ...\}}, that of a list is \texttt{[<value>, ...]}
  \item
    You can open files using one of the following: \\
    \quad \texttt{using open(<filename>, <access\_option>) as f:} \\
    \quad \texttt{~~~~<STUFF>} \\
    (preferred, since the file will be closed correctly if an error happens!) \\
    \quad \texttt{f = open(<filename>, <access\_option>); <STUFF>; close f}
  \item
    For reading/writing, you can use \texttt{f.read()}, \texttt{f.readlines()}, \texttt{f.write(<text>)}, and \texttt{f.writeline(<text>)}. Try out what the different versions do in an editor! 
  \end{itemize}
\end{frame}

\begin{frame}{Methods and Classes}
  \begin{enumerate}
  \item
    Write a method with which you can add more people to the room list
  \item
    Make a \texttt{class} person and personlist, which have the same contents as the previous dictionary and list
  \item
    Add the add-person method to the personlist class
  \item
    Also add a \texttt{\_\_str\_\_}-method to the person class, with which you can print the information contained in the class
  \end{enumerate}

  \textit{Hints:}
  \begin{itemize}
  \item
    To create methods and classes, you need to use \texttt{def <method>(<arg>, ...):} and \texttt{class <Class>:}
  \item
    Classes can be initialized with the magic method \texttt{def \_\_init\_\_(self, <arg>, ...)}
  \item
    If you assign \texttt{self.<name> = <value>}, the variable you assign can be used throughout the class
  \item
    \texttt{\_\_str\_\_} is another magic method without args (except \texttt{self}!), which should return a string and is automatically called by \texttt{print(<Class>)}
  \end{itemize}
\end{frame}

\begin{frame}{Libraries and Multiple Files}
  \begin{enumerate}
  \item
    Put the classes you just created in their own files and the main-method in another one
  \item
    Use \texttt{sys.argv} to add people to the personlist class from the commandline
  \item
    Check out python's \texttt{argparse} standard library and write some parameters for your executable
  \item
    Add methods to load and save your personlist as \texttt{json} and \texttt{yaml} files
  \end{enumerate}

  \textit{Hints:}
  \begin{itemize}
  \item
    You can make python files and libraries accessible to each other using \texttt{import}-statements
  \item
    Try out what \texttt{sys.argv} does with a simple macro if you are unsure
  \item
    The references for \href{https://docs.python.org/3/library/argparse.html}{\color{ugoelogodark}\uline{\texttt{argparse}}}, \href{https://docs.python.org/3/library/json.html}{\color{ugoelogodark}\uline{\texttt{json}}}, and \href{https://yaml.readthedocs.io/en/latest/}{\color{ugoelogodark}\uline{\texttt{yaml}}} can tell you how each of these packages works!
  \end{itemize}
\end{frame}

\begin{frame}{More Stuff to Try out}

  If you still have time left, try out one of these tasks:
  \begin{itemize}
  \item
    Use \texttt{numpy}-arrays to check out the anticommutators of the $\gamma$-matrices. Do they behave as expected?
  \item
    Build a class `Uncertainty' which allows adding in quadrature
  \item
    Expand your person-class to also include a subclass `Student' with inheritance. At this point, the argument `level-of-study' of course only makes sense for students.
  \end{itemize}

\end{frame}

\begin{frame}{Summary}
\begin{itemize}
  \item
    Overall, Python is a very user-friendly programming language
  \item
    Now used extensively for Data Science {\color{ugoelogodark}$\rightarrow$} Connection with C++ based \texttt{ROOT} via \texttt{uproot} module
    \begin{itemize}
      \item[$\hookrightarrow$]
        More about that tomorrow
    \end{itemize}
  \item
    Of course we could only give you a very brief overview of Python. If you are stuck during your thesis, just look online for help or ask your friendly PhD student!
\end{itemize}
\hrule height0pt
\vfill
\hrule height0pt
\begin{center}
  \textbf{\Large \color{ugoelogodark}Thanks for your attention}
\end{center}
\end{frame}

% \appendix

\end{document}
